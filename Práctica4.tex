
\documentclass[11pt]{article}
\usepackage[utf8]{inputenc}
\usepackage[english]{babel}
\usepackage{amsthm, amsmath}
\usepackage{nccmath} %Para centrar ecuaciones
\usepackage{graphicx}
\usepackage{enumitem}
\usepackage{algorithmic}
\usepackage{whilecode2}
\usepackage{verbatim}

\graphicspath{ {Images/} }
    \title{\textbf{Práctica 4}}
    \author{Juan Manuel Cardeñosa Borrego}
    \date{}
    
    \addtolength{\topmargin}{-3cm}
    \addtolength{\textheight}{3cm}
\begin{document}

\maketitle
\thispagestyle{empty}

\section*{Ejercicio 1}
El desarrollo del cálculo de la menor codificación del programa WHILE "diverger".
\\
En este ejercicio he planteado el siguiente código (con 0 argumentos):\\
\begin{whilecode}[H]

$X_1 \Assig X_1 + 1$\;
 \While{$X_3 \not = 0$}{

 \DefaultVar{1}\Assig\DefaultVar{1}

 }
\end{whilecode}

Además, ha obtenido en la codificación el número natural: 9678369754

\begin{figure}[htp]
\centering
\includegraphics[scale=0.70]{Ejercicio1.png}
\end{figure}


\section*{Ejercicio 2}
El código Octave que hace un print de todos los vectores, y una captura de ejemplo de ejecución.\\

\verbatiminput{allVector.m}

Ejemplo de ejecución en la siguiente hoja, como es un buble infinito, dado que los números naturales son infinitos, he realizado una captura de los 16 primeros.

\begin{figure}[htp]
\centering
\includegraphics[scale=0.60]{Ejercicio2.png}
\end{figure}

\newpage
\section*{Ejercicio 3}
El código Octave que hace un print de todos los programas WHILE, y una captura de ejemplo de ejecución.

\verbatiminput{allProgramWhile.m}

Como es un buble infinito, ya que hay tantos programas while como números naturales, es decir, infinitos, he realizado una captura de los 30 primeros.

\begin{figure}[htp]
\centering
\includegraphics[scale=0.60]{Ejercicio3.png}
\end{figure}


\end{document}

